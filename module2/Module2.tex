\documentclass[11pt]{article}

    \usepackage[breakable]{tcolorbox}
    \usepackage{parskip} % Stop auto-indenting (to mimic markdown behaviour)
    

    % Basic figure setup, for now with no caption control since it's done
    % automatically by Pandoc (which extracts ![](path) syntax from Markdown).
    \usepackage{graphicx}
    % Maintain compatibility with old templates. Remove in nbconvert 6.0
    \let\Oldincludegraphics\includegraphics
    % Ensure that by default, figures have no caption (until we provide a
    % proper Figure object with a Caption API and a way to capture that
    % in the conversion process - todo).
    \usepackage{caption}
    \DeclareCaptionFormat{nocaption}{}
    \captionsetup{format=nocaption,aboveskip=0pt,belowskip=0pt}

    \usepackage{float}
    \floatplacement{figure}{H} % forces figures to be placed at the correct location
    \usepackage{xcolor} % Allow colors to be defined
    \usepackage{enumerate} % Needed for markdown enumerations to work
    \usepackage{geometry} % Used to adjust the document margins
    \usepackage{amsmath} % Equations
    \usepackage{amssymb} % Equations
    \usepackage{textcomp} % defines textquotesingle
    % Hack from http://tex.stackexchange.com/a/47451/13684:
    \AtBeginDocument{%
        \def\PYZsq{\textquotesingle}% Upright quotes in Pygmentized code
    }
    \usepackage{upquote} % Upright quotes for verbatim code
    \usepackage{eurosym} % defines \euro

    \usepackage{iftex}
    \ifPDFTeX
        \usepackage[T1]{fontenc}
        \IfFileExists{alphabeta.sty}{
              \usepackage{alphabeta}
          }{
              \usepackage[mathletters]{ucs}
              \usepackage[utf8x]{inputenc}
          }
    \else
        \usepackage{fontspec}
        \usepackage{unicode-math}
    \fi

    \usepackage{fancyvrb} % verbatim replacement that allows latex
    \usepackage{grffile} % extends the file name processing of package graphics 
                         % to support a larger range
    \makeatletter % fix for old versions of grffile with XeLaTeX
    \@ifpackagelater{grffile}{2019/11/01}
    {
      % Do nothing on new versions
    }
    {
      \def\Gread@@xetex#1{%
        \IfFileExists{"\Gin@base".bb}%
        {\Gread@eps{\Gin@base.bb}}%
        {\Gread@@xetex@aux#1}%
      }
    }
    \makeatother
    \usepackage[Export]{adjustbox} % Used to constrain images to a maximum size
    \adjustboxset{max size={0.9\linewidth}{0.9\paperheight}}

    % The hyperref package gives us a pdf with properly built
    % internal navigation ('pdf bookmarks' for the table of contents,
    % internal cross-reference links, web links for URLs, etc.)
    \usepackage{hyperref}
    % The default LaTeX title has an obnoxious amount of whitespace. By default,
    % titling removes some of it. It also provides customization options.
    \usepackage{titling}
    \usepackage{longtable} % longtable support required by pandoc >1.10
    \usepackage{booktabs}  % table support for pandoc > 1.12.2
    \usepackage{array}     % table support for pandoc >= 2.11.3
    \usepackage{calc}      % table minipage width calculation for pandoc >= 2.11.1
    \usepackage[inline]{enumitem} % IRkernel/repr support (it uses the enumerate* environment)
    \usepackage[normalem]{ulem} % ulem is needed to support strikethroughs (\sout)
                                % normalem makes italics be italics, not underlines
    \usepackage{mathrsfs}
    

    
    % Colors for the hyperref package
    \definecolor{urlcolor}{rgb}{0,.145,.698}
    \definecolor{linkcolor}{rgb}{.71,0.21,0.01}
    \definecolor{citecolor}{rgb}{.12,.54,.11}

    % ANSI colors
    \definecolor{ansi-black}{HTML}{3E424D}
    \definecolor{ansi-black-intense}{HTML}{282C36}
    \definecolor{ansi-red}{HTML}{E75C58}
    \definecolor{ansi-red-intense}{HTML}{B22B31}
    \definecolor{ansi-green}{HTML}{00A250}
    \definecolor{ansi-green-intense}{HTML}{007427}
    \definecolor{ansi-yellow}{HTML}{DDB62B}
    \definecolor{ansi-yellow-intense}{HTML}{B27D12}
    \definecolor{ansi-blue}{HTML}{208FFB}
    \definecolor{ansi-blue-intense}{HTML}{0065CA}
    \definecolor{ansi-magenta}{HTML}{D160C4}
    \definecolor{ansi-magenta-intense}{HTML}{A03196}
    \definecolor{ansi-cyan}{HTML}{60C6C8}
    \definecolor{ansi-cyan-intense}{HTML}{258F8F}
    \definecolor{ansi-white}{HTML}{C5C1B4}
    \definecolor{ansi-white-intense}{HTML}{A1A6B2}
    \definecolor{ansi-default-inverse-fg}{HTML}{FFFFFF}
    \definecolor{ansi-default-inverse-bg}{HTML}{000000}

    % common color for the border for error outputs.
    \definecolor{outerrorbackground}{HTML}{FFDFDF}

    % commands and environments needed by pandoc snippets
    % extracted from the output of `pandoc -s`
    \providecommand{\tightlist}{%
      \setlength{\itemsep}{0pt}\setlength{\parskip}{0pt}}
    \DefineVerbatimEnvironment{Highlighting}{Verbatim}{commandchars=\\\{\}}
    % Add ',fontsize=\small' for more characters per line
    \newenvironment{Shaded}{}{}
    \newcommand{\KeywordTok}[1]{\textcolor[rgb]{0.00,0.44,0.13}{\textbf{{#1}}}}
    \newcommand{\DataTypeTok}[1]{\textcolor[rgb]{0.56,0.13,0.00}{{#1}}}
    \newcommand{\DecValTok}[1]{\textcolor[rgb]{0.25,0.63,0.44}{{#1}}}
    \newcommand{\BaseNTok}[1]{\textcolor[rgb]{0.25,0.63,0.44}{{#1}}}
    \newcommand{\FloatTok}[1]{\textcolor[rgb]{0.25,0.63,0.44}{{#1}}}
    \newcommand{\CharTok}[1]{\textcolor[rgb]{0.25,0.44,0.63}{{#1}}}
    \newcommand{\StringTok}[1]{\textcolor[rgb]{0.25,0.44,0.63}{{#1}}}
    \newcommand{\CommentTok}[1]{\textcolor[rgb]{0.38,0.63,0.69}{\textit{{#1}}}}
    \newcommand{\OtherTok}[1]{\textcolor[rgb]{0.00,0.44,0.13}{{#1}}}
    \newcommand{\AlertTok}[1]{\textcolor[rgb]{1.00,0.00,0.00}{\textbf{{#1}}}}
    \newcommand{\FunctionTok}[1]{\textcolor[rgb]{0.02,0.16,0.49}{{#1}}}
    \newcommand{\RegionMarkerTok}[1]{{#1}}
    \newcommand{\ErrorTok}[1]{\textcolor[rgb]{1.00,0.00,0.00}{\textbf{{#1}}}}
    \newcommand{\NormalTok}[1]{{#1}}
    
    % Additional commands for more recent versions of Pandoc
    \newcommand{\ConstantTok}[1]{\textcolor[rgb]{0.53,0.00,0.00}{{#1}}}
    \newcommand{\SpecialCharTok}[1]{\textcolor[rgb]{0.25,0.44,0.63}{{#1}}}
    \newcommand{\VerbatimStringTok}[1]{\textcolor[rgb]{0.25,0.44,0.63}{{#1}}}
    \newcommand{\SpecialStringTok}[1]{\textcolor[rgb]{0.73,0.40,0.53}{{#1}}}
    \newcommand{\ImportTok}[1]{{#1}}
    \newcommand{\DocumentationTok}[1]{\textcolor[rgb]{0.73,0.13,0.13}{\textit{{#1}}}}
    \newcommand{\AnnotationTok}[1]{\textcolor[rgb]{0.38,0.63,0.69}{\textbf{\textit{{#1}}}}}
    \newcommand{\CommentVarTok}[1]{\textcolor[rgb]{0.38,0.63,0.69}{\textbf{\textit{{#1}}}}}
    \newcommand{\VariableTok}[1]{\textcolor[rgb]{0.10,0.09,0.49}{{#1}}}
    \newcommand{\ControlFlowTok}[1]{\textcolor[rgb]{0.00,0.44,0.13}{\textbf{{#1}}}}
    \newcommand{\OperatorTok}[1]{\textcolor[rgb]{0.40,0.40,0.40}{{#1}}}
    \newcommand{\BuiltInTok}[1]{{#1}}
    \newcommand{\ExtensionTok}[1]{{#1}}
    \newcommand{\PreprocessorTok}[1]{\textcolor[rgb]{0.74,0.48,0.00}{{#1}}}
    \newcommand{\AttributeTok}[1]{\textcolor[rgb]{0.49,0.56,0.16}{{#1}}}
    \newcommand{\InformationTok}[1]{\textcolor[rgb]{0.38,0.63,0.69}{\textbf{\textit{{#1}}}}}
    \newcommand{\WarningTok}[1]{\textcolor[rgb]{0.38,0.63,0.69}{\textbf{\textit{{#1}}}}}
    
    
    % Define a nice break command that doesn't care if a line doesn't already
    % exist.
    \def\br{\hspace*{\fill} \\* }
    % Math Jax compatibility definitions
    \def\gt{>}
    \def\lt{<}
    \let\Oldtex\TeX
    \let\Oldlatex\LaTeX
    \renewcommand{\TeX}{\textrm{\Oldtex}}
    \renewcommand{\LaTeX}{\textrm{\Oldlatex}}
    % Document parameters
    % Document title
    \title{Module2}
    
    
    
    
    
% Pygments definitions
\makeatletter
\def\PY@reset{\let\PY@it=\relax \let\PY@bf=\relax%
    \let\PY@ul=\relax \let\PY@tc=\relax%
    \let\PY@bc=\relax \let\PY@ff=\relax}
\def\PY@tok#1{\csname PY@tok@#1\endcsname}
\def\PY@toks#1+{\ifx\relax#1\empty\else%
    \PY@tok{#1}\expandafter\PY@toks\fi}
\def\PY@do#1{\PY@bc{\PY@tc{\PY@ul{%
    \PY@it{\PY@bf{\PY@ff{#1}}}}}}}
\def\PY#1#2{\PY@reset\PY@toks#1+\relax+\PY@do{#2}}

\@namedef{PY@tok@w}{\def\PY@tc##1{\textcolor[rgb]{0.73,0.73,0.73}{##1}}}
\@namedef{PY@tok@c}{\let\PY@it=\textit\def\PY@tc##1{\textcolor[rgb]{0.24,0.48,0.48}{##1}}}
\@namedef{PY@tok@cp}{\def\PY@tc##1{\textcolor[rgb]{0.61,0.40,0.00}{##1}}}
\@namedef{PY@tok@k}{\let\PY@bf=\textbf\def\PY@tc##1{\textcolor[rgb]{0.00,0.50,0.00}{##1}}}
\@namedef{PY@tok@kp}{\def\PY@tc##1{\textcolor[rgb]{0.00,0.50,0.00}{##1}}}
\@namedef{PY@tok@kt}{\def\PY@tc##1{\textcolor[rgb]{0.69,0.00,0.25}{##1}}}
\@namedef{PY@tok@o}{\def\PY@tc##1{\textcolor[rgb]{0.40,0.40,0.40}{##1}}}
\@namedef{PY@tok@ow}{\let\PY@bf=\textbf\def\PY@tc##1{\textcolor[rgb]{0.67,0.13,1.00}{##1}}}
\@namedef{PY@tok@nb}{\def\PY@tc##1{\textcolor[rgb]{0.00,0.50,0.00}{##1}}}
\@namedef{PY@tok@nf}{\def\PY@tc##1{\textcolor[rgb]{0.00,0.00,1.00}{##1}}}
\@namedef{PY@tok@nc}{\let\PY@bf=\textbf\def\PY@tc##1{\textcolor[rgb]{0.00,0.00,1.00}{##1}}}
\@namedef{PY@tok@nn}{\let\PY@bf=\textbf\def\PY@tc##1{\textcolor[rgb]{0.00,0.00,1.00}{##1}}}
\@namedef{PY@tok@ne}{\let\PY@bf=\textbf\def\PY@tc##1{\textcolor[rgb]{0.80,0.25,0.22}{##1}}}
\@namedef{PY@tok@nv}{\def\PY@tc##1{\textcolor[rgb]{0.10,0.09,0.49}{##1}}}
\@namedef{PY@tok@no}{\def\PY@tc##1{\textcolor[rgb]{0.53,0.00,0.00}{##1}}}
\@namedef{PY@tok@nl}{\def\PY@tc##1{\textcolor[rgb]{0.46,0.46,0.00}{##1}}}
\@namedef{PY@tok@ni}{\let\PY@bf=\textbf\def\PY@tc##1{\textcolor[rgb]{0.44,0.44,0.44}{##1}}}
\@namedef{PY@tok@na}{\def\PY@tc##1{\textcolor[rgb]{0.41,0.47,0.13}{##1}}}
\@namedef{PY@tok@nt}{\let\PY@bf=\textbf\def\PY@tc##1{\textcolor[rgb]{0.00,0.50,0.00}{##1}}}
\@namedef{PY@tok@nd}{\def\PY@tc##1{\textcolor[rgb]{0.67,0.13,1.00}{##1}}}
\@namedef{PY@tok@s}{\def\PY@tc##1{\textcolor[rgb]{0.73,0.13,0.13}{##1}}}
\@namedef{PY@tok@sd}{\let\PY@it=\textit\def\PY@tc##1{\textcolor[rgb]{0.73,0.13,0.13}{##1}}}
\@namedef{PY@tok@si}{\let\PY@bf=\textbf\def\PY@tc##1{\textcolor[rgb]{0.64,0.35,0.47}{##1}}}
\@namedef{PY@tok@se}{\let\PY@bf=\textbf\def\PY@tc##1{\textcolor[rgb]{0.67,0.36,0.12}{##1}}}
\@namedef{PY@tok@sr}{\def\PY@tc##1{\textcolor[rgb]{0.64,0.35,0.47}{##1}}}
\@namedef{PY@tok@ss}{\def\PY@tc##1{\textcolor[rgb]{0.10,0.09,0.49}{##1}}}
\@namedef{PY@tok@sx}{\def\PY@tc##1{\textcolor[rgb]{0.00,0.50,0.00}{##1}}}
\@namedef{PY@tok@m}{\def\PY@tc##1{\textcolor[rgb]{0.40,0.40,0.40}{##1}}}
\@namedef{PY@tok@gh}{\let\PY@bf=\textbf\def\PY@tc##1{\textcolor[rgb]{0.00,0.00,0.50}{##1}}}
\@namedef{PY@tok@gu}{\let\PY@bf=\textbf\def\PY@tc##1{\textcolor[rgb]{0.50,0.00,0.50}{##1}}}
\@namedef{PY@tok@gd}{\def\PY@tc##1{\textcolor[rgb]{0.63,0.00,0.00}{##1}}}
\@namedef{PY@tok@gi}{\def\PY@tc##1{\textcolor[rgb]{0.00,0.52,0.00}{##1}}}
\@namedef{PY@tok@gr}{\def\PY@tc##1{\textcolor[rgb]{0.89,0.00,0.00}{##1}}}
\@namedef{PY@tok@ge}{\let\PY@it=\textit}
\@namedef{PY@tok@gs}{\let\PY@bf=\textbf}
\@namedef{PY@tok@gp}{\let\PY@bf=\textbf\def\PY@tc##1{\textcolor[rgb]{0.00,0.00,0.50}{##1}}}
\@namedef{PY@tok@go}{\def\PY@tc##1{\textcolor[rgb]{0.44,0.44,0.44}{##1}}}
\@namedef{PY@tok@gt}{\def\PY@tc##1{\textcolor[rgb]{0.00,0.27,0.87}{##1}}}
\@namedef{PY@tok@err}{\def\PY@bc##1{{\setlength{\fboxsep}{\string -\fboxrule}\fcolorbox[rgb]{1.00,0.00,0.00}{1,1,1}{\strut ##1}}}}
\@namedef{PY@tok@kc}{\let\PY@bf=\textbf\def\PY@tc##1{\textcolor[rgb]{0.00,0.50,0.00}{##1}}}
\@namedef{PY@tok@kd}{\let\PY@bf=\textbf\def\PY@tc##1{\textcolor[rgb]{0.00,0.50,0.00}{##1}}}
\@namedef{PY@tok@kn}{\let\PY@bf=\textbf\def\PY@tc##1{\textcolor[rgb]{0.00,0.50,0.00}{##1}}}
\@namedef{PY@tok@kr}{\let\PY@bf=\textbf\def\PY@tc##1{\textcolor[rgb]{0.00,0.50,0.00}{##1}}}
\@namedef{PY@tok@bp}{\def\PY@tc##1{\textcolor[rgb]{0.00,0.50,0.00}{##1}}}
\@namedef{PY@tok@fm}{\def\PY@tc##1{\textcolor[rgb]{0.00,0.00,1.00}{##1}}}
\@namedef{PY@tok@vc}{\def\PY@tc##1{\textcolor[rgb]{0.10,0.09,0.49}{##1}}}
\@namedef{PY@tok@vg}{\def\PY@tc##1{\textcolor[rgb]{0.10,0.09,0.49}{##1}}}
\@namedef{PY@tok@vi}{\def\PY@tc##1{\textcolor[rgb]{0.10,0.09,0.49}{##1}}}
\@namedef{PY@tok@vm}{\def\PY@tc##1{\textcolor[rgb]{0.10,0.09,0.49}{##1}}}
\@namedef{PY@tok@sa}{\def\PY@tc##1{\textcolor[rgb]{0.73,0.13,0.13}{##1}}}
\@namedef{PY@tok@sb}{\def\PY@tc##1{\textcolor[rgb]{0.73,0.13,0.13}{##1}}}
\@namedef{PY@tok@sc}{\def\PY@tc##1{\textcolor[rgb]{0.73,0.13,0.13}{##1}}}
\@namedef{PY@tok@dl}{\def\PY@tc##1{\textcolor[rgb]{0.73,0.13,0.13}{##1}}}
\@namedef{PY@tok@s2}{\def\PY@tc##1{\textcolor[rgb]{0.73,0.13,0.13}{##1}}}
\@namedef{PY@tok@sh}{\def\PY@tc##1{\textcolor[rgb]{0.73,0.13,0.13}{##1}}}
\@namedef{PY@tok@s1}{\def\PY@tc##1{\textcolor[rgb]{0.73,0.13,0.13}{##1}}}
\@namedef{PY@tok@mb}{\def\PY@tc##1{\textcolor[rgb]{0.40,0.40,0.40}{##1}}}
\@namedef{PY@tok@mf}{\def\PY@tc##1{\textcolor[rgb]{0.40,0.40,0.40}{##1}}}
\@namedef{PY@tok@mh}{\def\PY@tc##1{\textcolor[rgb]{0.40,0.40,0.40}{##1}}}
\@namedef{PY@tok@mi}{\def\PY@tc##1{\textcolor[rgb]{0.40,0.40,0.40}{##1}}}
\@namedef{PY@tok@il}{\def\PY@tc##1{\textcolor[rgb]{0.40,0.40,0.40}{##1}}}
\@namedef{PY@tok@mo}{\def\PY@tc##1{\textcolor[rgb]{0.40,0.40,0.40}{##1}}}
\@namedef{PY@tok@ch}{\let\PY@it=\textit\def\PY@tc##1{\textcolor[rgb]{0.24,0.48,0.48}{##1}}}
\@namedef{PY@tok@cm}{\let\PY@it=\textit\def\PY@tc##1{\textcolor[rgb]{0.24,0.48,0.48}{##1}}}
\@namedef{PY@tok@cpf}{\let\PY@it=\textit\def\PY@tc##1{\textcolor[rgb]{0.24,0.48,0.48}{##1}}}
\@namedef{PY@tok@c1}{\let\PY@it=\textit\def\PY@tc##1{\textcolor[rgb]{0.24,0.48,0.48}{##1}}}
\@namedef{PY@tok@cs}{\let\PY@it=\textit\def\PY@tc##1{\textcolor[rgb]{0.24,0.48,0.48}{##1}}}

\def\PYZbs{\char`\\}
\def\PYZus{\char`\_}
\def\PYZob{\char`\{}
\def\PYZcb{\char`\}}
\def\PYZca{\char`\^}
\def\PYZam{\char`\&}
\def\PYZlt{\char`\<}
\def\PYZgt{\char`\>}
\def\PYZsh{\char`\#}
\def\PYZpc{\char`\%}
\def\PYZdl{\char`\$}
\def\PYZhy{\char`\-}
\def\PYZsq{\char`\'}
\def\PYZdq{\char`\"}
\def\PYZti{\char`\~}
% for compatibility with earlier versions
\def\PYZat{@}
\def\PYZlb{[}
\def\PYZrb{]}
\makeatother


    % For linebreaks inside Verbatim environment from package fancyvrb. 
    \makeatletter
        \newbox\Wrappedcontinuationbox 
        \newbox\Wrappedvisiblespacebox 
        \newcommand*\Wrappedvisiblespace {\textcolor{red}{\textvisiblespace}} 
        \newcommand*\Wrappedcontinuationsymbol {\textcolor{red}{\llap{\tiny$\m@th\hookrightarrow$}}} 
        \newcommand*\Wrappedcontinuationindent {3ex } 
        \newcommand*\Wrappedafterbreak {\kern\Wrappedcontinuationindent\copy\Wrappedcontinuationbox} 
        % Take advantage of the already applied Pygments mark-up to insert 
        % potential linebreaks for TeX processing. 
        %        {, <, #, %, $, ' and ": go to next line. 
        %        _, }, ^, &, >, - and ~: stay at end of broken line. 
        % Use of \textquotesingle for straight quote. 
        \newcommand*\Wrappedbreaksatspecials {% 
            \def\PYGZus{\discretionary{\char`\_}{\Wrappedafterbreak}{\char`\_}}% 
            \def\PYGZob{\discretionary{}{\Wrappedafterbreak\char`\{}{\char`\{}}% 
            \def\PYGZcb{\discretionary{\char`\}}{\Wrappedafterbreak}{\char`\}}}% 
            \def\PYGZca{\discretionary{\char`\^}{\Wrappedafterbreak}{\char`\^}}% 
            \def\PYGZam{\discretionary{\char`\&}{\Wrappedafterbreak}{\char`\&}}% 
            \def\PYGZlt{\discretionary{}{\Wrappedafterbreak\char`\<}{\char`\<}}% 
            \def\PYGZgt{\discretionary{\char`\>}{\Wrappedafterbreak}{\char`\>}}% 
            \def\PYGZsh{\discretionary{}{\Wrappedafterbreak\char`\#}{\char`\#}}% 
            \def\PYGZpc{\discretionary{}{\Wrappedafterbreak\char`\%}{\char`\%}}% 
            \def\PYGZdl{\discretionary{}{\Wrappedafterbreak\char`\$}{\char`\$}}% 
            \def\PYGZhy{\discretionary{\char`\-}{\Wrappedafterbreak}{\char`\-}}% 
            \def\PYGZsq{\discretionary{}{\Wrappedafterbreak\textquotesingle}{\textquotesingle}}% 
            \def\PYGZdq{\discretionary{}{\Wrappedafterbreak\char`\"}{\char`\"}}% 
            \def\PYGZti{\discretionary{\char`\~}{\Wrappedafterbreak}{\char`\~}}% 
        } 
        % Some characters . , ; ? ! / are not pygmentized. 
        % This macro makes them "active" and they will insert potential linebreaks 
        \newcommand*\Wrappedbreaksatpunct {% 
            \lccode`\~`\.\lowercase{\def~}{\discretionary{\hbox{\char`\.}}{\Wrappedafterbreak}{\hbox{\char`\.}}}% 
            \lccode`\~`\,\lowercase{\def~}{\discretionary{\hbox{\char`\,}}{\Wrappedafterbreak}{\hbox{\char`\,}}}% 
            \lccode`\~`\;\lowercase{\def~}{\discretionary{\hbox{\char`\;}}{\Wrappedafterbreak}{\hbox{\char`\;}}}% 
            \lccode`\~`\:\lowercase{\def~}{\discretionary{\hbox{\char`\:}}{\Wrappedafterbreak}{\hbox{\char`\:}}}% 
            \lccode`\~`\?\lowercase{\def~}{\discretionary{\hbox{\char`\?}}{\Wrappedafterbreak}{\hbox{\char`\?}}}% 
            \lccode`\~`\!\lowercase{\def~}{\discretionary{\hbox{\char`\!}}{\Wrappedafterbreak}{\hbox{\char`\!}}}% 
            \lccode`\~`\/\lowercase{\def~}{\discretionary{\hbox{\char`\/}}{\Wrappedafterbreak}{\hbox{\char`\/}}}% 
            \catcode`\.\active
            \catcode`\,\active 
            \catcode`\;\active
            \catcode`\:\active
            \catcode`\?\active
            \catcode`\!\active
            \catcode`\/\active 
            \lccode`\~`\~ 	
        }
    \makeatother

    \let\OriginalVerbatim=\Verbatim
    \makeatletter
    \renewcommand{\Verbatim}[1][1]{%
        %\parskip\z@skip
        \sbox\Wrappedcontinuationbox {\Wrappedcontinuationsymbol}%
        \sbox\Wrappedvisiblespacebox {\FV@SetupFont\Wrappedvisiblespace}%
        \def\FancyVerbFormatLine ##1{\hsize\linewidth
            \vtop{\raggedright\hyphenpenalty\z@\exhyphenpenalty\z@
                \doublehyphendemerits\z@\finalhyphendemerits\z@
                \strut ##1\strut}%
        }%
        % If the linebreak is at a space, the latter will be displayed as visible
        % space at end of first line, and a continuation symbol starts next line.
        % Stretch/shrink are however usually zero for typewriter font.
        \def\FV@Space {%
            \nobreak\hskip\z@ plus\fontdimen3\font minus\fontdimen4\font
            \discretionary{\copy\Wrappedvisiblespacebox}{\Wrappedafterbreak}
            {\kern\fontdimen2\font}%
        }%
        
        % Allow breaks at special characters using \PYG... macros.
        \Wrappedbreaksatspecials
        % Breaks at punctuation characters . , ; ? ! and / need catcode=\active 	
        \OriginalVerbatim[#1,codes*=\Wrappedbreaksatpunct]%
    }
    \makeatother

    % Exact colors from NB
    \definecolor{incolor}{HTML}{303F9F}
    \definecolor{outcolor}{HTML}{D84315}
    \definecolor{cellborder}{HTML}{CFCFCF}
    \definecolor{cellbackground}{HTML}{F7F7F7}
    
    % prompt
    \makeatletter
    \newcommand{\boxspacing}{\kern\kvtcb@left@rule\kern\kvtcb@boxsep}
    \makeatother
    \newcommand{\prompt}[4]{
        {\ttfamily\llap{{\color{#2}[#3]:\hspace{3pt}#4}}\vspace{-\baselineskip}}
    }
    

    
    % Prevent overflowing lines due to hard-to-break entities
    \sloppy 
    % Setup hyperref package
    \hypersetup{
      breaklinks=true,  % so long urls are correctly broken across lines
      colorlinks=true,
      urlcolor=urlcolor,
      linkcolor=linkcolor,
      citecolor=citecolor,
      }
    % Slightly bigger margins than the latex defaults
    
    \geometry{verbose,tmargin=1in,bmargin=1in,lmargin=1in,rmargin=1in}
    
    

\begin{document}
    
    \maketitle
    
    

    
    
    
    
    \begin{Verbatim}[commandchars=\\\{\}]
<IPython.core.display.HTML object>
    \end{Verbatim}

    
    For module 2 we'll be looking at techniques for dealing with big data.
In particular binning strategies and the datashader library (which
possibly proves we'll never need to bin large data for visualization
ever again.)

To demonstrate these concepts we'll be looking at the PLUTO dataset put
out by New York City's department of city planning. PLUTO contains data
about every tax lot in New York City.

PLUTO data can be downloaded from
\href{https://www1.nyc.gov/site/planning/data-maps/open-data/dwn-pluto-mappluto.page}{here}.
Unzip them to the same directory as this notebook, and you should be
able to read them in using this (or very similar) code. Also take note
of the data dictionary, it'll come in handy for this assignment.

    \begin{Verbatim}[commandchars=\\\{\}]
/var/folders/f4/v17stl257nn9w40qtyt496380000gn/T/ipykernel\_24847/1232822874.py:1
1: DtypeWarning:

Columns (21,22,24,26,28) have mixed types. Specify dtype option on import or set
low\_memory=False.

    \end{Verbatim}

    I'll also do some prep for the geographic component of this data, which
we'll be relying on for datashader.

You're not required to know how I'm retrieving the lattitude and
longitude here, but for those interested: this dataset uses a flat x-y
projection (assuming for a small enough area that the world is flat for
easier calculations), and this needs to be projected back to traditional
lattitude and longitude.

    \hypertarget{part-1-binning-and-aggregation}{%
\subsection{Part 1: Binning and
Aggregation}\label{part-1-binning-and-aggregation}}

Binning is a common strategy for visualizing large datasets. Binning is
inherent to a few types of visualizations, such as histograms and
\href{https://plot.ly/python/2D-Histogram/}{2D histograms} (also check
out their close relatives:
\href{https://plot.ly/python/2d-density-plots/}{2D density plots} and
the more general form:
\href{https://plot.ly/python/heatmaps/}{heatmaps}.

While these visualization types explicitly include binning, any type of
visualization used with aggregated data can be looked at in the same
way. For example, lets say we wanted to look at building construction
over time. This would be best viewed as a line graph, but we can still
think of our results as being binned by year:

    
    
    Something looks off\ldots{} You're going to have to deal with this
imperfect data to answer this first question.

But first: some notes on pandas. Pandas dataframes are a different beast
than R dataframes, here are some tips to help you get up to speed:

\begin{center}\rule{0.5\linewidth}{0.5pt}\end{center}

Hello all, here are some pandas tips to help you guys through this
homework:

\href{https://pandas.pydata.org/pandas-docs/stable/indexing.html}{Indexing
and Selecting}: .loc and .iloc are the analogs for base R subsetting, or
filter() in dplyr

\href{https://pandas.pydata.org/pandas-docs/stable/groupby.html}{Group
By}: This is the pandas analog to group\_by() and the appended function
the analog to summarize(). Try out a few examples of this, and display
the results in Jupyter. Take note of what's happening to the indexes,
you'll notice that they'll become hierarchical. I personally find this
more of a burden than a help, and this sort of hierarchical indexing
leads to a fundamentally different experience compared to R dataframes.
Once you perform an aggregation, try running the resulting hierarchical
datafrome through a
\href{https://pandas.pydata.org/pandas-docs/stable/generated/pandas.DataFrame.reset_index.html}{reset\_index()}.

\href{https://pandas.pydata.org/pandas-docs/stable/generated/pandas.DataFrame.reset_index.html}{Reset\_index}:
I personally find the hierarchical indexes more of a burden than a help,
and this sort of hierarchical indexing leads to a fundamentally
different experience compared to R dataframes. reset\_index() is a way
of restoring a dataframe to a flatter index style. Grouping is where
you'll notice it the most, but it's also useful when you filter data,
and in a few other split-apply-combine workflows. With pandas indexes
are more meaningful, so use this if you start getting unexpected
results.

Indexes are more important in Pandas than in R. If you delve deeper into
the using python for data science, you'll begin to see the benefits in
many places (despite the personal gripes I highlighted above.) One place
these indexes come in handy is with time series data. The pandas docs
have a
\href{http://pandas.pydata.org/pandas-docs/stable/timeseries.html}{huge
section} on datetime indexing. In particular, check out
\href{https://pandas.pydata.org/pandas-docs/stable/generated/pandas.DataFrame.resample.html}{resample},
which provides time series specific aggregation.

\href{https://pandas.pydata.org/pandas-docs/stable/merging.html}{Merging,
joining, and concatenation}: There's some overlap between these
different types of merges, so use this as your guide. Concat is a single
function that replaces cbind and rbind in R, and the results are driven
by the indexes. Read through these examples to get a feel on how these
are performed, but you will have to manage your indexes when you're
using these functions. Merges are fairly similar to merges in R,
similarly mapping to SQL joins.

Apply: This is explained in the ``group by'' section linked above. These
are your analogs to the plyr library in R. Take note of the lambda
syntax used here, these are anonymous functions in python. Rather than
predefining a custom function, you can just define it inline using
lambda.

Browse through the other sections for some other specifics, in
particular reshaping and categorical data (pandas' answer to factors.)
Pandas can take a while to get used to, but it is a pretty strong
framework that makes more advanced functions easier once you get used to
it. Rolling functions for example follow logically from the apply
workflow (and led to the best google results ever when I first tried to
find this out and googled ``pandas rolling'')

Google Wes Mckinney's book ``Python for Data Analysis,'' which is a
cookbook style intro to pandas. It's an O'Reilly book that should be
pretty available out there.

\begin{center}\rule{0.5\linewidth}{0.5pt}\end{center}

\hypertarget{question}{%
\subsubsection{Question}\label{question}}

After a few building collapses, the City of New York is going to begin
investigating older buildings for safety. The city is particularly
worried about buildings that were unusually tall when they were built,
since best-practices for safety hadn't yet been determined. Create a
graph that shows how many buildings of a certain number of floors were
built in each year (note: you may want to use a log scale for the number
of buildings). Find a strategy to bin buildings (It should be clear
20-29-story buildings, 30-39-story buildings, and 40-49-story buildings
were first built in large numbers, but does it make sense to continue in
this way as you get taller?)

\begin{center}\rule{0.5\linewidth}{0.5pt}\end{center}

\hypertarget{answer}{%
\subsubsection{Answer}\label{answer}}

First we can check what the range of values for numfloors is:

    \begin{Verbatim}[commandchars=\\\{\}]
min      1.0
max    104.0
Name: numfloors, dtype: float64
    \end{Verbatim}

    Then we can visualize the distribution of numfloors using a histogram:

    
    
    We can see the vast majority of buildings have 10 or fewer floors, but
there are a few outliers. We can use a logarithmic scale to better
visualize the distribution of the data:

    
    
    From the early plot of buildings built by year it seems that certain
years had few or no buildings built. Let's check the number of buildings
built in each year using a subset of the data:

    \begin{Verbatim}[commandchars=\\\{\}]
1900.0     6440
1901.0    22230
1902.0      476
1903.0      460
1904.0      522
1905.0     6914
1906.0     1026
1907.0      996
1908.0      839
1909.0     1540
1910.0    41983
1911.0     1083
1912.0      883
1913.0      732
1914.0      668
1915.0    15363
1916.0      687
1917.0      534
1918.0      291
1919.0      369
1920.0    87703
1921.0     1118
1922.0     1246
1923.0     1641
1924.0     2246
1925.0    69677
1926.0     3256
1927.0     3598
1928.0     4346
1929.0     2189
1930.0    74907
1931.0    31000
1932.0     1224
1933.0      946
1934.0      374
1935.0    25085
1936.0      643
1937.0      701
1938.0      764
1939.0      929
1940.0    37904
Name: yearbuilt, dtype: int64
    \end{Verbatim}

    We can see while there are no years with 0 buildings built in this range
there is a lot of variance in the data. For example 1933 only has 946
buildings built, while 1931 has 31,000. This gives us good reason to
aggregate the data by decade.

Now we can create a chart that shows the number of buildings built in
each decade by number of floors. We will bin using the min and max floor
numbers in increments of 10. We will also make sure to use the log scale
for the y-axis to better visualize the data.

    
    
    It is clear from the plot above that even using a logarithmic scale, the
1-20 floor buildings are dominating the plot. Also, the line scatter
chart format gives us a lot of noise in the form of data resolution that
we do not need in order to visualize the data. Let's filter out
buildings with less than 20 floors and replot using stacked bar charts:

    
    
    \hypertarget{part-2-datashader}{%
\subsection{Part 2: Datashader}\label{part-2-datashader}}

Datashader is a library from Anaconda that does away with the need for
binning data. It takes in all of your datapoints, and based on the
canvas and range returns a pixel-by-pixel calculations to come up with
the best representation of the data. In short, this completely
eliminates the need for binning your data.

As an example, lets continue with our question above and look at a 2D
histogram of YearBuilt vs NumFloors:

    
    
    This shows us the distribution, but it's subject to some biases
discussed in the Anaconda notebook
\href{https://anaconda.org/jbednar/plotting_pitfalls/notebook}{Plotting
Perils}.

Here is what the same plot would look like in datashader:
 
            
    
    \begin{center}
    \adjustimage{max size={0.9\linewidth}{0.9\paperheight}}{Module2_files/Module2_23_0.png}
    \end{center}
    { \hspace*{\fill} \\}
    

    That's technically just a scatterplot, but the points are smartly placed
and colored to mimic what one gets in a heatmap. Based on the pixel
size, it will either display individual points, or will color the points
of denser regions.

Datashader really shines when looking at geographic information. Here
are the latitudes and longitudes of our dataset plotted out, giving us a
map of the city colored by density of structures:
 
            
    
    \begin{center}
    \adjustimage{max size={0.9\linewidth}{0.9\paperheight}}{Module2_files/Module2_25_0.png}
    \end{center}
    { \hspace*{\fill} \\}
    

    Interestingly, since we're looking at structures, the large buildings of
Manhattan show up as less dense on the map. The densest areas measured
by number of lots would be single or multi family townhomes.

Unfortunately, Datashader doesn't have the best documentation. Browse
through the examples from their
\href{https://github.com/bokeh/datashader/tree/master/examples}{github
repo}. I would focus on the
\href{https://anaconda.org/jbednar/pipeline/notebook}{visualization
pipeline} and the \href{https://anaconda.org/jbednar/census/notebook}{US
Census} Example for the question below. Feel free to use my samples as
templates as well when you work on this problem.

\hypertarget{question}{%
\subsubsection{Question}\label{question}}

You work for a real estate developer and are researching underbuilt
areas of the city. After looking in the
\href{https://www1.nyc.gov/assets/planning/download/pdf/data-maps/open-data/pluto_datadictionary.pdf?v=17v1_1}{Pluto
data dictionary}, you've discovered that all tax assessments consist of
two parts: The assessment of the land and assessment of the structure.
You reason that there should be a correlation between these two values:
more valuable land will have more valuable structures on them (more
valuable in this case refers not just to a mansion vs a bungalow, but an
apartment tower vs a single family home). Deviations from the norm could
represent underbuilt or overbuilt areas of the city. You also recently
read a really cool blog post about
\href{http://www.joshuastevens.net/cartography/make-a-bivariate-choropleth-map/}{bivariate
choropleth maps}, and think the technique could be used for this
problem.

Datashader is really cool, but it's not that great at labeling your
visualization. Don't worry about providing a legend, but provide a quick
explanation as to which areas of the city are overbuilt, which areas are
underbuilt, and which areas are built in a way that's properly
correlated with their land value.

\begin{center}\rule{0.5\linewidth}{0.5pt}\end{center}

\hypertarget{answer}{%
\subsubsection{Answer}\label{answer}}

There is no separated value of the structural assessment in the dataset
but we have the total and land assessment values so we can subtract the
land value from the total and get what we assume to be the structural
assessment value.

            \begin{tcolorbox}[breakable, size=fbox, boxrule=.5pt, pad at break*=1mm, opacityfill=0]
\begin{Verbatim}[commandchars=\\\{\}]
  borough  block  lot     cd    bct2020     bctcb2020  ct2010  cb2010  \textbackslash{}
0      SI   1597  125  502.0  5029104.0  5.029104e+10  291.04  3007.0
2      BK   4794    1  309.0  3080600.0  3.080600e+10  806.00  2000.0
3      BK   1488  105  303.0  3037500.0  3.037500e+10  375.00  1001.0
4      BK   4794   17  309.0  3080600.0  3.080600e+10  806.00  2000.0
5      BK   4794   78  309.0  3080600.0  3.080600e+10  806.00  2000.0

   schooldist  council  {\ldots}  plutomapid firm07\_flag  pfirm15\_flag  version  \textbackslash{}
0        31.0     50.0  {\ldots}           1         NaN           NaN     22v2
2        17.0     41.0  {\ldots}           1         NaN           NaN     22v2
3        16.0     41.0  {\ldots}           1         NaN           NaN     22v2
4        17.0     41.0  {\ldots}           1         NaN           NaN     22v2
5        17.0     41.0  {\ldots}           1         NaN           NaN     22v2

   dcpedited   latitude  longitude notes  decade assessstruct
0        NaN  40.611140 -74.164376   NaN  1960.0      46020.0
2        NaN  40.661794 -73.942532   NaN  1890.0     308700.0
3        NaN  40.686484 -73.920169   NaN  1990.0      40740.0
4        NaN  40.661859 -73.941991   NaN  1990.0      46380.0
5        NaN  40.661517 -73.942539   NaN  2000.0      78480.0

[5 rows x 94 columns]
\end{Verbatim}
\end{tcolorbox}
        
    It wasn't clear to me how to generate the color palettes, so I used this
tool \url{https://colorbrewer2.org/\#type=diverging\&scheme=PiYG\&n=9/}

    Going to use \textbf{Quantile} segmentation for the data which because
we are segmenting into 3 distinct groups means simply dividing the data
set into three parts and labeling it.

    \begin{Verbatim}[commandchars=\\\{\}]
[0.0, 11580.0]
[11580.0, 18120.0]
[18120.0, 3205633833.0]
    \end{Verbatim}

            \begin{tcolorbox}[breakable, size=fbox, boxrule=.5pt, pad at break*=1mm, opacityfill=0]
\begin{Verbatim}[commandchars=\\\{\}]
  borough  block  lot     cd    bct2020     bctcb2020  ct2010  cb2010  \textbackslash{}
0      SI   1597  125  502.0  5029104.0  5.029104e+10  291.04  3007.0
2      BK   4794    1  309.0  3080600.0  3.080600e+10  806.00  2000.0
3      BK   1488  105  303.0  3037500.0  3.037500e+10  375.00  1001.0
4      BK   4794   17  309.0  3080600.0  3.080600e+10  806.00  2000.0
5      BK   4794   78  309.0  3080600.0  3.080600e+10  806.00  2000.0

   schooldist  council  {\ldots}  firm07\_flag pfirm15\_flag  version  dcpedited  \textbackslash{}
0        31.0     50.0  {\ldots}          NaN          NaN     22v2        NaN
2        17.0     41.0  {\ldots}          NaN          NaN     22v2        NaN
3        16.0     41.0  {\ldots}          NaN          NaN     22v2        NaN
4        17.0     41.0  {\ldots}          NaN          NaN     22v2        NaN
5        17.0     41.0  {\ldots}          NaN          NaN     22v2        NaN

    latitude  longitude  notes  decade assessstruct colorOne
0  40.611140 -74.164376    NaN  1960.0      46020.0        1
2  40.661794 -73.942532    NaN  1890.0     308700.0        1
3  40.686484 -73.920169    NaN  1990.0      40740.0        2
4  40.661859 -73.941991    NaN  1990.0      46380.0        3
5  40.661517 -73.942539    NaN  2000.0      78480.0        3

[5 rows x 95 columns]
\end{Verbatim}
\end{tcolorbox}
        
    \begin{Verbatim}[commandchars=\\\{\}]
[0.0, 33300.0]
[33300.0, 60000.0]
[60000.0, 4343286717.0]
    \end{Verbatim}

            \begin{tcolorbox}[breakable, size=fbox, boxrule=.5pt, pad at break*=1mm, opacityfill=0]
\begin{Verbatim}[commandchars=\\\{\}]
  borough  block  lot     cd    bct2020     bctcb2020  ct2010  cb2010  \textbackslash{}
0      SI   1597  125  502.0  5029104.0  5.029104e+10  291.04  3007.0
2      BK   4794    1  309.0  3080600.0  3.080600e+10  806.00  2000.0
3      BK   1488  105  303.0  3037500.0  3.037500e+10  375.00  1001.0
4      BK   4794   17  309.0  3080600.0  3.080600e+10  806.00  2000.0
5      BK   4794   78  309.0  3080600.0  3.080600e+10  806.00  2000.0

   schooldist  council  {\ldots}  pfirm15\_flag version  dcpedited   latitude  \textbackslash{}
0        31.0     50.0  {\ldots}           NaN    22v2        NaN  40.611140
2        17.0     41.0  {\ldots}           NaN    22v2        NaN  40.661794
3        16.0     41.0  {\ldots}           NaN    22v2        NaN  40.686484
4        17.0     41.0  {\ldots}           NaN    22v2        NaN  40.661859
5        17.0     41.0  {\ldots}           NaN    22v2        NaN  40.661517

   longitude  notes  decade assessstruct colorOne colorTwo
0 -74.164376    NaN  1960.0      46020.0        1        B
2 -73.942532    NaN  1890.0     308700.0        1        C
3 -73.920169    NaN  1990.0      40740.0        2        B
4 -73.941991    NaN  1990.0      46380.0        3        B
5 -73.942539    NaN  2000.0      78480.0        3        C

[5 rows x 96 columns]
\end{Verbatim}
\end{tcolorbox}
        
            \begin{tcolorbox}[breakable, size=fbox, boxrule=.5pt, pad at break*=1mm, opacityfill=0]
\begin{Verbatim}[commandchars=\\\{\}]
  borough  block  lot     cd    bct2020     bctcb2020  ct2010  cb2010  \textbackslash{}
0      SI   1597  125  502.0  5029104.0  5.029104e+10  291.04  3007.0
2      BK   4794    1  309.0  3080600.0  3.080600e+10  806.00  2000.0
3      BK   1488  105  303.0  3037500.0  3.037500e+10  375.00  1001.0
4      BK   4794   17  309.0  3080600.0  3.080600e+10  806.00  2000.0
5      BK   4794   78  309.0  3080600.0  3.080600e+10  806.00  2000.0

   schooldist  council  {\ldots}  version dcpedited   latitude  longitude  notes  \textbackslash{}
0        31.0     50.0  {\ldots}     22v2       NaN  40.611140 -74.164376    NaN
2        17.0     41.0  {\ldots}     22v2       NaN  40.661794 -73.942532    NaN
3        16.0     41.0  {\ldots}     22v2       NaN  40.686484 -73.920169    NaN
4        17.0     41.0  {\ldots}     22v2       NaN  40.661859 -73.941991    NaN
5        17.0     41.0  {\ldots}     22v2       NaN  40.661517 -73.942539    NaN

   decade  assessstruct colorOne colorTwo combined
0  1960.0       46020.0        1        B       1B
2  1890.0      308700.0        1        C       1C
3  1990.0       40740.0        2        B       2B
4  1990.0       46380.0        3        B       3B
5  2000.0       78480.0        3        C       3C

[5 rows x 97 columns]
\end{Verbatim}
\end{tcolorbox}
         
            
    
    \begin{center}
    \adjustimage{max size={0.9\linewidth}{0.9\paperheight}}{Module2_files/Module2_37_0.png}
    \end{center}
    { \hspace*{\fill} \\}
    


    % Add a bibliography block to the postdoc
    
    
    
\end{document}
